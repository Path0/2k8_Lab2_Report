%Template generated by Ben Manning
%Purdue University
%btmannin@purdue.edu
%Last modified: 7/7/2021


\documentclass[notitlepage, 12pt]{report}  %The document class will setup a lot of basic formatting.  Report class will start with justifying paragraphs setup your different types of sections.

%Different packages allow you to add more functions that will make your time in LaTeX easier.
\usepackage{amsmath}
\usepackage{graphicx}
\usepackage{caption}
\usepackage{url}
\usepackage{circuitikz} % circuit drawer

%\usepackage{biblatex} %Imports biblatex package
\usepackage[style=numeric]{biblatex}
\addbibresource{bib.bib}

\usepackage[top=2cm, bottom=2cm, left=2cm, right=2cm]{geometry}

\title{Experiment 2 Report}

\begin{document}
%Everything needs to begin and end.  


\begin{center}
\large \textbf{Experiment X Report} \\ %\large and \small can help make text sizes vary throughout your document.
%\textbf will bold the text that is in the curly brackets
\small 
Andrew Lykken\\
Anna Kishnani\\
26 January 2023\\
Section 004 (Abraham Yakisan)\\
%\rule{500pt}{.1pt} 

\end{center}

% space between title and abstract
\vspace{4in}


\begin{abstract}
abstract here 
\end{abstract}

\newpage

\section*{Task 1} %Each task has a section including (but not limited to) Objective, 
% Procedure, Results / Calculations, Conclusions


\subsection*{Objective}

\subsection*{Procedure}

\subsection*{Results / Calculations}


\subsection*{Conclusions}




\newpage

\printbibliography[title={\Large References}] %Prints out the bibliography sources that you have used in the document.

\end{document}